\begin{table}[ht]
\centering
\begin{tabular}{rllllllll}
  \hline
 & Name & Title & Description & Version & m.Date & m.Time & Author & Citation \\ 
  \hline
1 & ExampleData.ModelOutput & Example data (TL curve) simulated with parameter set from Pagonis 2007 & Example data (TL curve) simulated with parameter set from Pagonis 2007 & 0.1.1
 &  &  & Johannes Friedrich, University of Bayreuth (Germany)$<$br /$>$ &  \\ 
  2 & extract\_parameters2FME & Prepare parameters for use with R package FME and function  fit\_RLumModel2data & Prepare parameters for use with R package FME and function  fit\_RLumModel2data & 0.1.1 & [2016-05-24] & (2017-10-17 & Johannes Friedrich, University of Bayreuth (Germany),$<$br /$>$ &  \\ 
  3 & fit\_RLumModel2data & Fit model parameters to experimental data & Fit model parameters to experimental data & 0.1.0 & [2016-04-29] & (2017-10-17 & Johannes Friedrich, University of Bayreuth (Germany)$<$br /$>$ &  \\ 
  4 & model\_LuminescenceSignals & Model Luminescence Signals & This function models luminescence signals for quartz based on published physical models. It is possible to simulate TL, (CW-) OSL, RF measurements in a arbitrary sequence. This sequence is definded as a  list  of certain abrivations. Furthermore it is possible to load a sequence direct from the Riso Sequence Editor. The output is an  RLum.Analysis object and so the plots are done by the  plot\_RLum.Analysis  function. If a SAR sequence is simulated the plot output can be disabled and SAR analyse functions can be used. & 0.1.4 & 2017-10-17 & 13:48:47
 & Johannes Friedrich, University of Bayreuth (Germany),$<$br /$>$ Sebastian Kreutzer, IRAMAT-CRP2A, Universite Bordeaux Montaigne (France)$<$br /$>$ &  \\ 
  5 & read\_SEQ2R & Parse a Risoe SEQ-file to a sequence neccessary for simulating quartz luminescence & A SEQ-file created by the Risoe Sequence Editor can be imported to simulate the sequence written in the sequence editor. & 0.1.0 & 2017-10-13 & 13:46:59
 & Johannes Friedrich, University of Bayreuth (Germany),$<$br /$>$ &  \\ 
   \hline
\end{tabular}
\end{table}

