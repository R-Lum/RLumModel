% latex table generated in R 3.3.0 by xtable 1.8-2 package
% Tue May 24 15:45:26 2016
\begin{table}[ht]
\centering
\begin{tabular}{rlllllll}
  \hline
 & Name & Title & Description & Version & m.Date & m.Time & Author \\ 
  \hline
1 & ExampleData.FittingTL & Example data with TL curves extracted from a TL-SAR protocol (lab code BT1195) & $<$br /$>$ Example data with TL curves extracted from a TL-SAR protocol (lab code BT1195)$<$br /$>$ & 0.1.0
 &  &  & $<$br /$>$ Johannes Friedrich, University of Bayreuth (Germany)$<$br /$>$ \\ 
  2 & ExampleData.ModelOutput & Example data (TL curve) simulated with parameter set from Pagonis 2007 & $<$br /$>$ Example data (TL curve) simulated with parameter set from Pagonis 2007$<$br /$>$ & 0.1.1
 &  &  & $<$br /$>$ Johannes Friedrich, University of Bayreuth (Germany)$<$br /$>$ \\ 
  3 & extract\_pars2FME & Prepare parameters for use with R package FME and function  fit\_RLumModel2data & $<$br /$>$ Prepare parameters for use with R package FME and function  fit\_RLumModel2data $<$br /$>$ & 0.1.0 & [2016-04-29]
 &  & $<$br /$>$ Johannes Friedrich, University of Bayreuth (Germany),$<$br /$>$ \\ 
  4 & fit\_RLumModel2data & Fit model parameters to experimental data & $<$br /$>$ Fit model parameters to experimental data$<$br /$>$ & 0.1.0 & [2016-04-29]
 &  & $<$br /$>$ Johannes Friedrich, University of Bayreuth (Germany)$<$br /$>$ \\ 
  5 & model\_LuminescenceSignals & Model Luminescence Signals & $<$br /$>$ This function models luminescence signals for quartz based on published physical models.$<$br /$>$ It is possible to simulate TL, (CW-) OSL, RF measurements in a arbitrary sequence. This$<$br /$>$ sequence is definded as a  list  of certain abrivations. Furthermore it is possible to$<$br /$>$ load a sequence direct from the Riso Sequence Editor.$<$br /$>$ The output is an  RLum.Analysis object and so the plots are done$<$br /$>$ by the  plot\_RLum.Analysis  function. If a SAR sequence is simulated the plot output can be disabled and SAR analyse functions$<$br /$>$ can be used.$<$br /$>$ & 0.1.2 & [2016-04-28]
 &  & $<$br /$>$ Johannes Friedrich, University of Bayreuth (Germany),$<$br /$>$ Sebastian Kreutzer, IRAMAT-CRP2A, Universite Bordeaux Montaigne (France)$<$br /$>$ \\ 
  6 & read\_SEQ2R & Parse a Risoe SEQ-file to a sequence neccessary for simulating quartz luminescence & $<$br /$>$ A SEQ-file created by the Risoe Sequence Editor can be imported to simulate the sequence written$<$br /$>$ in the sequence editor.$<$br /$>$ & 0.1.0
 &  &  & $<$br /$>$ Johannes Friedrich, University of Bayreuth (Germany),$<$br /$>$ \\ 
   \hline
\end{tabular}
\end{table}

